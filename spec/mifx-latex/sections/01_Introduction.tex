\section{Introduction}

The Machining Intent Format eXchange (MIFX) defines a portable, system-neutral job package for the structured exchange of machining intent between software systems.

MIFX provides a structured representation of:

\begin{itemize}[noitemsep]
    \item machining setups,
    \item ordered operations,
    \item tool references and definitions,
    \item and optional machine-neutral motion data (e.g., APT/CL) for reference.
\end{itemize}

The format is designed to reduce ambiguity and information loss in machining job exchange by providing a structured, machine-readable representation of machining intent.

MIFX does not prescribe how machining is executed, validated, simulated, or governed. It does not define machining strategies, machine capabilities, or execution logic. It does not replace CAM systems, verification tools, machine controllers, or lifecycle management platforms. It provides a portable envelope for machining intent.

A MIFX package is distributed as a single archive file with the extension \texttt{.mifx}. The archive contains:

\begin{itemize}[noitemsep]
    \item a required intent definition file (\texttt{job.json}),
    \item setup definitions (\texttt{setup/}),
    \item operation definitions (\texttt{operations/}),
    \item optional motion data (\texttt{cldata/}),
    \item optional geometry (\texttt{geometry/}),
    \item and optional tool payload definitions (\texttt{geometry/tooling/}).
\end{itemize}

The file \texttt{job.json} is the sole required entry point of a valid MIFX package. It references all other components using relative paths within the archive.

All additional components are optional and may be ignored by conforming consumers. Conformance to MIFX is defined solely by adherence to the published JSON specification and structural rules defined in this document.

MIFX is an open specification intended for unrestricted implementation.