\section{Extension Policy}

MIFX is designed with a minimal normative core and an unrestricted
extension mechanism.

Extensions enable customization without altering, redefining,
or fragmenting the core data model.

\subsection{Principle}

The MIFX core specification is normative and stable.

Extensions are:

\begin{itemize}
    \item OPTIONAL,
    \item non-normative,
    \item implementation-defined.
\end{itemize}

A conforming MIFX document SHALL remain valid and interpretable
even if all extensions are ignored.

Core validity SHALL NOT depend on the presence,
interpretation, or processing of extensions.

\subsection{Goals of the Extension Mechanism}

The extension system exists to:

\begin{itemize}
    \item allow vendor-specific metadata,
    \item support controller-specific requirements,
    \item enable enterprise customization,
    \item accommodate future domain expansion,
    \item prevent core specification inflation.
\end{itemize}

Extensions SHALL NOT modify, override, reinterpret,
or supersede the normative semantics of the MIFX core model.

\subsection{Extension Declaration}

A document MAY declare extension namespaces using
the \texttt{ExtensionList} element.

\begin{verbatim}
<ExtensionList>
  <Extension ns="vendor-name" path="extensions/vendor-name/" />
</ExtensionList>
\end{verbatim}

The \texttt{ns} attribute defines a logical namespace identifier.

The \texttt{path} attribute defines the relative directory
within the MIFX package where extension-specific resources reside.

Each namespace value SHALL be unique within the document.

Consumers that do not recognize a declared namespace
SHALL ignore it without error.

Unknown elements within declared extension namespaces
SHALL be ignored unless explicitly required by that extension.

\subsection{Inline Extensions}

Inline extensions MAY be declared within:

\begin{itemize}
    \item \texttt{Tool} elements
    \item \texttt{Operation} elements
\end{itemize}

Example:

\begin{verbatim}
<Extensions>
    <Param name="vendor.parameter" value="..."/>
</Extensions>
\end{verbatim}

Inline extensions are informational and non-normative.

They SHALL NOT override, redefine, contradict,
or reinterpret core-defined attributes.

\subsection{External Extension Resources}

A MIFX package MAY include an \texttt{extensions/} directory.

Files within this directory:

\begin{itemize}
    \item are OPTIONAL,
    \item MAY follow vendor-defined schemas,
    \item are outside the scope of this specification.
\end{itemize}

The presence or absence of extension files
SHALL NOT affect the validity of the MIFX core document.

\subsection{Non-Interference Clause}

Extensions SHALL NOT:

\begin{itemize}
    \item alter document order semantics,
    \item redefine core element meaning,
    \item introduce hidden structural dependencies,
    \item require mandatory external services,
    \item introduce required execution semantics.
\end{itemize}

Ignoring extensions SHALL NOT cause structural ambiguity
or invalidate core interpretation.

\subsection{Stability Guarantee}

The existence of extensions SHALL NOT force revision
of the MIFX core specification.

Core evolution, if any, SHALL remain independent
of vendor-specific or domain-specific extension mechanisms.