\section{Extension Policy}

MIFX is designed with a minimal normative core and an unrestricted
extension mechanism.

Extensions enable customization without altering, redefining,
or fragmenting the core data model.

\subsection{Principle}

The MIFX core specification is normative and stable.

Extensions are:

\begin{itemize}
    \item OPTIONAL,
    \item non-normative,
    \item implementation-defined.
\end{itemize}

A conforming MIFX document SHALL remain valid and interpretable
even if all extensions are ignored.

Core validity SHALL NOT depend on the presence,
interpretation, or processing of extensions.

\subsection{Goals of the Extension Mechanism}

The extension system exists to:

\begin{itemize}
    \item allow vendor-specific metadata,
    \item support controller-specific requirements,
    \item enable enterprise customization,
    \item accommodate future domain expansion,
    \item prevent core specification inflation.
\end{itemize}

Extensions SHALL NOT modify, override, reinterpret,
or supersede the normative semantics of the MIFX core model.

\subsection{Extension Mechanism}

Extensions MAY be introduced as additional JSON fields within:

\begin{itemize}
    \item \texttt{job.json},
    \item Setup documents,
    \item Operation documents,
    \item Tool definitions.
\end{itemize}

Extension fields SHALL NOT conflict with core-defined property names.

Extension properties SHOULD use vendor-qualified naming conventions
(e.g., \texttt{"vendorName.property"} or nested vendor objects)
to reduce collision risk.

Consumers that do not recognize an extension field
SHALL ignore it without error.

Unknown fields SHALL NOT invalidate structural parsing.

\subsection{Inline Extensions}

Inline extensions MAY appear as additional JSON properties
alongside core-defined properties.

Example:

\begin{verbatim}
{
  "id": "op-1",
  "setupRef": "set-1",
  "vendorX.customParameter": "value"
}
\end{verbatim}

Inline extensions are informational and non-normative.

They SHALL NOT override, redefine, contradict,
or reinterpret core-defined properties.

\subsection{External Extension Resources}

A MIFX package MAY include additional files
that are not referenced by the core model.

Such files:

\begin{itemize}
    \item are OPTIONAL,
    \item MAY follow vendor-defined schemas,
    \item are outside the scope of this specification.
\end{itemize}

The presence or absence of such files
SHALL NOT affect the validity of the MIFX core structure.

\subsection{Non-Interference Clause}

Extensions SHALL NOT:

\begin{itemize}
    \item alter declared Setup or Operation ordering,
    \item redefine core property meaning,
    \item introduce hidden structural dependencies,
    \item require mandatory external services,
    \item introduce required execution semantics.
\end{itemize}

Ignoring extensions SHALL NOT cause structural ambiguity
or invalidate core interpretation.

\subsection{Stability Guarantee}

The existence of extensions SHALL NOT force revision
of the MIFX core specification.

Core evolution, if any, SHALL remain independent
of vendor-specific or domain-specific extension mechanisms.