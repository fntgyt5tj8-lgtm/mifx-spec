\section{MIFX Core Data Model}

This section defines the normative structure of the \texttt{job.xml} document.

The document is strictly sequential.
Document order defines semantic order.
Elements SHALL appear in the defined order.
Consumers SHALL treat document order as semantically significant.
No execution graph, dependency model, or behavioral semantics are defined.

The following structural components are REQUIRED:

\begin{itemize}
    \item Root \texttt{Job} element
    \item \texttt{Header} element
    \item Global unit definition
    \item \texttt{OperationList} element
\end{itemize}

All other elements are OPTIONAL unless explicitly stated.

\subsection{Root Element}

The root element SHALL be:

\begin{verbatim}
<Job exportVersion="x.y">
    ...
</Job>
\end{verbatim}

The attribute \texttt{exportVersion} identifies the schema revision.

Only one \texttt{Job} element is permitted per document.

\subsection{Root Structure}

The \texttt{Job} element SHALL contain elements in the following order:

\begin{enumerate}
    \item ExtensionList (optional)
    \item Header (required)
    \item SetupList (optional)
    \item ToolList (optional)
    \item OperationList (required)
\end{enumerate}

If present, \texttt{ExtensionList} SHALL appear immediately after
the opening \texttt{Job} element and before \texttt{Header}.

The absence of \texttt{ExtensionList} implies no declared namespaces.

\subsection{ExtensionList Declaration}

A MIFX document MAY declare extension namespaces
using an \texttt{ExtensionList} element.

\begin{verbatim}
<ExtensionList>
  <Extension ns="acme-cmm" path="extensions/acme-cmm/" />
  <Extension ns="autodesk" path="extensions/autodesk/" />
</ExtensionList>
\end{verbatim}

The \texttt{ns} attribute defines a logical namespace identifier.
Each namespace value SHALL be unique within the document.

The \texttt{path} attribute defines the relative directory
within the MIFX package where extension-specific resources reside.

Extension namespaces SHALL NOT redefine, override,
or alter the normative semantics of the MIFX core data model.

Consumers that do not recognize a declared namespace
SHALL ignore it without error.

Core validity SHALL NOT depend upon extension-defined namespaces.

\subsection{Header}

The \texttt{Header} element contains provenance and job metadata.

\begin{verbatim}
<Header>
    <Export id="UUID" at="ISO8601" by="System" version="x.y"/>
    <Job>
        <Name>...</Name>
        <Units system="metric" linear="mm"/>
    </Job>
    <Source>
        <CAM>...</CAM>
        <Design>...</Design>
        <Document>...</Document>
    </Source>
</Header>
\end{verbatim}

The \texttt{system} attribute SHALL be either \texttt{metric} or \texttt{imperial}.
The \texttt{linear} attribute SHALL be either \texttt{mm} or \texttt{in}.

\textbf{Export} identifies the generating system instance and timestamp.
\textbf{Units} define the global linear unit system for the document.
\textbf{Source} is informational and non-normative.

\subsection{SetupList}

Setups define workholding configurations and coordinate transforms.

\begin{verbatim}
<SetupList>
    <Setup index="1" id="set-1">
        ...
    </Setup>
</SetupList>
\end{verbatim}

Setups SHALL be interpreted sequentially by document order.
The \texttt{id} attribute SHALL be unique within the document.

\subsubsection{Setup Transform}

Each Setup MAY define a 4x4 transformation matrix.

\begin{verbatim}
<Transform type="matrix4x4"
           unit="mm"
           convention="local_to_world"
           storage="row-major">
    <Row>...</Row>
    <Row>...</Row>
    <Row>...</Row>
    <Row>...</Row>
</Transform>
\end{verbatim}

The matrix defines the Setup coordinate frame relative to the Job world space.
If omitted, identity transformation SHALL be assumed.

\subsection{ToolList}

Tools define identity and minimal descriptive metadata.

\begin{verbatim}
<ToolList>
    <Tool id="main:1" group="main">
        ...
    </Tool>
</ToolList>
\end{verbatim}

Tool IDs SHALL be unique within the document.

\subsubsection{Tool Extensions}

\begin{verbatim}
<Extensions>
    <Param name="holderVendor" value="Maritool" level="display"/>
    <Param name="calculation_tolerance" value="0.001" level="system"/>
</Extensions>
\end{verbatim}

The \texttt{level} attribute is REQUIRED on every \texttt{Param}
and SHALL be one of:

\begin{itemize}
    \item \texttt{level="display"} --- intended for human-facing display
    \item \texttt{level="system"} --- intended for integration or downstream processing
\end{itemize}

Extension parameters are non-normative metadata.
They SHALL NOT alter core document semantics.
Consumers MAY ignore extension parameters regardless of \texttt{level}.

\subsection{OperationList}

Operations define machining intent units.

\begin{verbatim}
<OperationList>
    <Operation id="op-1"
               setupRef="set-1"
               toolRef="main:1">
        ...
    </Operation>
</OperationList>
\end{verbatim}

Operations SHALL be interpreted sequentially by document order.
The \texttt{id} attribute SHALL be unique within the document.

If present:
\begin{itemize}
    \item \texttt{setupRef} SHALL reference a valid Setup id.
    \item \texttt{toolRef} SHALL reference a valid Tool id.
\end{itemize}

\subsubsection{Artifact References}

Operations MAY reference zero or more external artifacts using one or more
\texttt{ArtifactRef} elements.

\begin{verbatim}
<ArtifactRef role="toolpath"
             kind="apt"
             stem="op-1"
             present="true"/>
\end{verbatim}

Artifact references are structured pointers to files contained within the MIFX package.

\paragraph{Resolution Rules}

The directory SHALL be determined by the \texttt{role} value:

\begin{itemize}
    \item \texttt{toolpath} → \texttt{cldata/}
    \item \texttt{in\_process\_payload} → \texttt{geometry/in\_process/}
    \item \texttt{tooling\_payload} → \texttt{geometry/tooling/}
\end{itemize}

The resolved file path SHALL be constructed as:

\begin{verbatim}
<directory>/<stem>.<kind>
\end{verbatim}

All directory names defined by this specification SHALL be treated as case-sensitive.

If \texttt{present="true"}, the referenced file SHALL exist within the package.
If \texttt{present="false"} or omitted, the file MAY be absent.

Artifact references SHALL NOT alter core structural semantics.

Additional roles MAY exist via extensions,
but SHALL NOT redefine or override core semantics.
Core validity SHALL NOT depend upon extension-defined roles.

\paragraph{Toolpath Artifact}

For \texttt{role="toolpath"}:

\begin{itemize}
    \item \texttt{kind} SHALL be either \texttt{apt} or \texttt{cl}
    \item Files SHALL reside in the \texttt{cldata/} directory
\end{itemize}

Toolpath artifacts SHALL conform to ISO 4343 APT/CL
or a documented equivalent neutral motion format.

Toolpath artifacts are non-normative with respect to execution.
MIFX does not define controller dialects or post-processing behavior.

\subsubsection{CuttingConditions}

\begin{verbatim}
<CuttingConditions>
    <SpindleDir>CW</SpindleDir>
    <Spindle unit="rpm">...</Spindle>
    <Feed unit="mm/min">...</Feed>
    <Coolant>through_tool</Coolant>
</CuttingConditions>
\end{verbatim}

These values represent declared intent only and do not define controller behavior.

\texttt{SpindleDir} SHALL be either \texttt{CW} or \texttt{CCW}.
\texttt{Coolant} MAY be \texttt{off}, \texttt{flood}, \texttt{mist},
\texttt{through\_tool}, \texttt{air}, or \texttt{flood\_tool}.

No inheritance model is defined.

\subsection{Determinism}

A conforming MIFX consumer SHALL be able to:

\begin{itemize}
    \item parse the XML document,
    \item reconstruct setup order,
    \item reconstruct operation order,
    \item resolve tool and setup references,
    \item interpret transformation matrices deterministically.
\end{itemize}

A conforming consumer SHALL produce identical structural interpretation
given identical document input.

Consumers SHALL NOT infer implicit relationships beyond those explicitly defined.

MIFX defines structure and relationships only.
MIFX does not define execution behavior, simulation,
controller syntax, or lifecycle management.